%
% API Documentation for PyHal
% Module hal.maths.maths
%
% Generated by epydoc 3.0.1
% [Sun Oct 22 12:02:07 2017]
%

%%%%%%%%%%%%%%%%%%%%%%%%%%%%%%%%%%%%%%%%%%%%%%%%%%%%%%%%%%%%%%%%%%%%%%%%%%%
%%                          Module Description                           %%
%%%%%%%%%%%%%%%%%%%%%%%%%%%%%%%%%%%%%%%%%%%%%%%%%%%%%%%%%%%%%%%%%%%%%%%%%%%

    \index{hal \textit{(package)}!hal.maths \textit{(package)}!hal.maths.maths \textit{(module)}|(}
\section{Module hal.maths.maths}

    \label{hal:maths:maths}
\begin{alltt}
A few elegant and powerful mathematical functions. 
\end{alltt}


%%%%%%%%%%%%%%%%%%%%%%%%%%%%%%%%%%%%%%%%%%%%%%%%%%%%%%%%%%%%%%%%%%%%%%%%%%%
%%                               Functions                               %%
%%%%%%%%%%%%%%%%%%%%%%%%%%%%%%%%%%%%%%%%%%%%%%%%%%%%%%%%%%%%%%%%%%%%%%%%%%%

  \subsection{Functions}

    \label{hal:maths:maths:get_prime}
    \index{hal \textit{(package)}!hal.maths \textit{(package)}!hal.maths.maths \textit{(module)}!hal.maths.maths.get\_prime \textit{(function)}}

    \vspace{0.5ex}

\hspace{.8\funcindent}\begin{boxedminipage}{\funcwidth}

    \raggedright \textbf{get\_prime}(\textit{bits})

    \vspace{-1.5ex}

    \rule{\textwidth}{0.5\fboxrule}
\setlength{\parskip}{2ex}
\begin{alltt}

:param bits: size of number to generate (bits)
:return: prime number of given size
\end{alltt}

\setlength{\parskip}{1ex}
    \end{boxedminipage}

    \label{hal:maths:maths:blum_blum_shub}
    \index{hal \textit{(package)}!hal.maths \textit{(package)}!hal.maths.maths \textit{(module)}!hal.maths.maths.blum\_blum\_shub \textit{(function)}}

    \vspace{0.5ex}

\hspace{.8\funcindent}\begin{boxedminipage}{\funcwidth}

    \raggedright \textbf{blum\_blum\_shub}(\textit{seed}, \textit{amount}, \textit{prime0}, \textit{prime1})

    \vspace{-1.5ex}

    \rule{\textwidth}{0.5\fboxrule}
\setlength{\parskip}{2ex}
\begin{alltt}

:param seed: seeder
:param amount: amount of number to generate
:param prime0: one prime number
:param prime1: the second prime number
:return: pseudo-number generator
\end{alltt}

\setlength{\parskip}{1ex}
    \end{boxedminipage}


%%%%%%%%%%%%%%%%%%%%%%%%%%%%%%%%%%%%%%%%%%%%%%%%%%%%%%%%%%%%%%%%%%%%%%%%%%%
%%                               Variables                               %%
%%%%%%%%%%%%%%%%%%%%%%%%%%%%%%%%%%%%%%%%%%%%%%%%%%%%%%%%%%%%%%%%%%%%%%%%%%%

  \subsection{Variables}

    \vspace{-1cm}
\hspace{\varindent}\begin{longtable}{|p{\varnamewidth}|p{\vardescrwidth}|l}
\cline{1-2}
\cline{1-2} \centering \textbf{Name} & \centering \textbf{Description}& \\
\cline{1-2}
\endhead\cline{1-2}\multicolumn{3}{r}{\small\textit{continued on next page}}\\\endfoot\cline{1-2}
\endlastfoot\raggedright L\-O\-W\-\_\-P\-R\-I\-M\-E\-S\- & \raggedright \textbf{Value:} 
{\tt \texttt{[}2\texttt{, }3\texttt{, }5\texttt{, }7\texttt{, }11\texttt{, }13\texttt{, }17\texttt{, }19\texttt{, }23\texttt{, }29\texttt{, }31\texttt{, }37\texttt{, }41\texttt{, }43\texttt{, }47\texttt{, }\texttt{...}}&\\
\cline{1-2}
\raggedright \_\-\_\-p\-a\-c\-k\-a\-g\-e\-\_\-\_\- & \raggedright \textbf{Value:} 
{\tt \texttt{'}\texttt{hal.maths}\texttt{'}}&\\
\cline{1-2}
\end{longtable}


%%%%%%%%%%%%%%%%%%%%%%%%%%%%%%%%%%%%%%%%%%%%%%%%%%%%%%%%%%%%%%%%%%%%%%%%%%%
%%                           Class Description                           %%
%%%%%%%%%%%%%%%%%%%%%%%%%%%%%%%%%%%%%%%%%%%%%%%%%%%%%%%%%%%%%%%%%%%%%%%%%%%

    \index{hal \textit{(package)}!hal.maths \textit{(package)}!hal.maths.maths \textit{(module)}!hal.maths.maths.Integer \textit{(class)}|(}
\subsection{Class Integer}

    \label{hal:maths:maths:Integer}
\begin{tabular}{cccccc}
% Line for object, linespec=[False]
\multicolumn{2}{r}{\settowidth{\BCL}{object}\multirow{2}{\BCL}{object}}
&&
  \\\cline{3-3}
  &&\multicolumn{1}{c|}{}
&&
  \\
&&\multicolumn{2}{l}{\textbf{hal.maths.maths.Integer}}
\end{tabular}

\begin{alltt}
Big int std python won't recognize 
\end{alltt}


%%%%%%%%%%%%%%%%%%%%%%%%%%%%%%%%%%%%%%%%%%%%%%%%%%%%%%%%%%%%%%%%%%%%%%%%%%%
%%                                Methods                                %%
%%%%%%%%%%%%%%%%%%%%%%%%%%%%%%%%%%%%%%%%%%%%%%%%%%%%%%%%%%%%%%%%%%%%%%%%%%%

  \subsubsection{Methods}

    \vspace{0.5ex}

\hspace{.8\funcindent}\begin{boxedminipage}{\funcwidth}

    \raggedright \textbf{\_\_init\_\_}(\textit{self}, \textit{string})

\setlength{\parskip}{2ex}
\begin{alltt}
x.\_\_init\_\_(...) initializes x; see help(type(x)) for signature
\end{alltt}

\setlength{\parskip}{1ex}
      Overrides: object.\_\_init\_\_ 	extit{(inherited documentation)}

    \end{boxedminipage}

    \label{hal:maths:maths:Integer:is_naive_prime}
    \index{hal \textit{(package)}!hal.maths \textit{(package)}!hal.maths.maths \textit{(module)}!hal.maths.maths.Integer \textit{(class)}!hal.maths.maths.Integer.is\_naive\_prime \textit{(method)}}

    \vspace{0.5ex}

\hspace{.8\funcindent}\begin{boxedminipage}{\funcwidth}

    \raggedright \textbf{is\_naive\_prime}(\textit{self})

    \vspace{-1.5ex}

    \rule{\textwidth}{0.5\fboxrule}
\setlength{\parskip}{2ex}
\begin{alltt}

:return: bool
    Checks if prime in very naive way
\end{alltt}

\setlength{\parskip}{1ex}
    \end{boxedminipage}

    \label{hal:maths:maths:Integer:is_probably_prime}
    \index{hal \textit{(package)}!hal.maths \textit{(package)}!hal.maths.maths \textit{(module)}!hal.maths.maths.Integer \textit{(class)}!hal.maths.maths.Integer.is\_probably\_prime \textit{(method)}}

    \vspace{0.5ex}

\hspace{.8\funcindent}\begin{boxedminipage}{\funcwidth}

    \raggedright \textbf{is\_probably\_prime}(\textit{self})

    \vspace{-1.5ex}

    \rule{\textwidth}{0.5\fboxrule}
\setlength{\parskip}{2ex}
\begin{alltt}

:return: test with miller-rabin
\end{alltt}

\setlength{\parskip}{1ex}
    \end{boxedminipage}

    \label{hal:maths:maths:Integer:test_miller_rabin}
    \index{hal \textit{(package)}!hal.maths \textit{(package)}!hal.maths.maths \textit{(module)}!hal.maths.maths.Integer \textit{(class)}!hal.maths.maths.Integer.test\_miller\_rabin \textit{(method)}}

    \vspace{0.5ex}

\hspace{.8\funcindent}\begin{boxedminipage}{\funcwidth}

    \raggedright \textbf{test\_miller\_rabin}(\textit{self}, \textit{precision})

    \vspace{-1.5ex}

    \rule{\textwidth}{0.5\fboxrule}
\setlength{\parskip}{2ex}
\begin{alltt}

:param precision: number of rounds to perform (higher -{\textgreater} better
precision)
:return: True iff probably prime
\end{alltt}

\setlength{\parskip}{1ex}
    \end{boxedminipage}


\large{\textbf{\textit{Inherited from object}}}

\begin{quote}
\_\_delattr\_\_(), \_\_format\_\_(), \_\_getattribute\_\_(), \_\_hash\_\_(), \_\_new\_\_(), \_\_reduce\_\_(), \_\_reduce\_ex\_\_(), \_\_repr\_\_(), \_\_setattr\_\_(), \_\_sizeof\_\_(), \_\_str\_\_(), \_\_subclasshook\_\_()
\end{quote}

%%%%%%%%%%%%%%%%%%%%%%%%%%%%%%%%%%%%%%%%%%%%%%%%%%%%%%%%%%%%%%%%%%%%%%%%%%%
%%                              Properties                               %%
%%%%%%%%%%%%%%%%%%%%%%%%%%%%%%%%%%%%%%%%%%%%%%%%%%%%%%%%%%%%%%%%%%%%%%%%%%%

  \subsubsection{Properties}

    \vspace{-1cm}
\hspace{\varindent}\begin{longtable}{|p{\varnamewidth}|p{\vardescrwidth}|l}
\cline{1-2}
\cline{1-2} \centering \textbf{Name} & \centering \textbf{Description}& \\
\cline{1-2}
\endhead\cline{1-2}\multicolumn{3}{r}{\small\textit{continued on next page}}\\\endfoot\cline{1-2}
\endlastfoot\multicolumn{2}{|l|}{\textit{Inherited from object}}\\
\multicolumn{2}{|p{\varwidth}|}{\raggedright \_\_class\_\_}\\
\cline{1-2}
\end{longtable}

    \index{hal \textit{(package)}!hal.maths \textit{(package)}!hal.maths.maths \textit{(module)}!hal.maths.maths.Integer \textit{(class)}|)}

%%%%%%%%%%%%%%%%%%%%%%%%%%%%%%%%%%%%%%%%%%%%%%%%%%%%%%%%%%%%%%%%%%%%%%%%%%%
%%                           Class Description                           %%
%%%%%%%%%%%%%%%%%%%%%%%%%%%%%%%%%%%%%%%%%%%%%%%%%%%%%%%%%%%%%%%%%%%%%%%%%%%

    \index{hal \textit{(package)}!hal.maths \textit{(package)}!hal.maths.maths \textit{(module)}!hal.maths.maths.EightQueen \textit{(class)}|(}
\subsection{Class EightQueen}

    \label{hal:maths:maths:EightQueen}
\begin{tabular}{cccccc}
% Line for object, linespec=[False]
\multicolumn{2}{r}{\settowidth{\BCL}{object}\multirow{2}{\BCL}{object}}
&&
  \\\cline{3-3}
  &&\multicolumn{1}{c|}{}
&&
  \\
&&\multicolumn{2}{l}{\textbf{hal.maths.maths.EightQueen}}
\end{tabular}

\begin{alltt}
8 queen problem solver 
\end{alltt}


%%%%%%%%%%%%%%%%%%%%%%%%%%%%%%%%%%%%%%%%%%%%%%%%%%%%%%%%%%%%%%%%%%%%%%%%%%%
%%                                Methods                                %%
%%%%%%%%%%%%%%%%%%%%%%%%%%%%%%%%%%%%%%%%%%%%%%%%%%%%%%%%%%%%%%%%%%%%%%%%%%%

  \subsubsection{Methods}

    \vspace{0.5ex}

\hspace{.8\funcindent}\begin{boxedminipage}{\funcwidth}

    \raggedright \textbf{\_\_init\_\_}(\textit{self}, \textit{board\_size})

\setlength{\parskip}{2ex}
\begin{alltt}
x.\_\_init\_\_(...) initializes x; see help(type(x)) for signature
\end{alltt}

\setlength{\parskip}{1ex}
      Overrides: object.\_\_init\_\_ 	extit{(inherited documentation)}

    \end{boxedminipage}

    \label{hal:maths:maths:EightQueen:under_attack}
    \index{hal \textit{(package)}!hal.maths \textit{(package)}!hal.maths.maths \textit{(module)}!hal.maths.maths.EightQueen \textit{(class)}!hal.maths.maths.EightQueen.under\_attack \textit{(static method)}}

    \vspace{0.5ex}

\hspace{.8\funcindent}\begin{boxedminipage}{\funcwidth}

    \raggedright \textbf{under\_attack}(\textit{col}, \textit{queens})

    \vspace{-1.5ex}

    \rule{\textwidth}{0.5\fboxrule}
\setlength{\parskip}{2ex}
\begin{alltt}

:param col: int
    Column number
:param queens: []
    List of queens
:return: bool
    True iff queen is under attack
\end{alltt}

\setlength{\parskip}{1ex}
    \end{boxedminipage}

    \label{hal:maths:maths:EightQueen:solve}
    \index{hal \textit{(package)}!hal.maths \textit{(package)}!hal.maths.maths \textit{(module)}!hal.maths.maths.EightQueen \textit{(class)}!hal.maths.maths.EightQueen.solve \textit{(method)}}

    \vspace{0.5ex}

\hspace{.8\funcindent}\begin{boxedminipage}{\funcwidth}

    \raggedright \textbf{solve}(\textit{self}, \textit{table\_size})

    \vspace{-1.5ex}

    \rule{\textwidth}{0.5\fboxrule}
\setlength{\parskip}{2ex}
\begin{alltt}

:param table\_size: int
    Size of table
:return: []
    List of possible solutions
\end{alltt}

\setlength{\parskip}{1ex}
    \end{boxedminipage}


\large{\textbf{\textit{Inherited from object}}}

\begin{quote}
\_\_delattr\_\_(), \_\_format\_\_(), \_\_getattribute\_\_(), \_\_hash\_\_(), \_\_new\_\_(), \_\_reduce\_\_(), \_\_reduce\_ex\_\_(), \_\_repr\_\_(), \_\_setattr\_\_(), \_\_sizeof\_\_(), \_\_str\_\_(), \_\_subclasshook\_\_()
\end{quote}

%%%%%%%%%%%%%%%%%%%%%%%%%%%%%%%%%%%%%%%%%%%%%%%%%%%%%%%%%%%%%%%%%%%%%%%%%%%
%%                              Properties                               %%
%%%%%%%%%%%%%%%%%%%%%%%%%%%%%%%%%%%%%%%%%%%%%%%%%%%%%%%%%%%%%%%%%%%%%%%%%%%

  \subsubsection{Properties}

    \vspace{-1cm}
\hspace{\varindent}\begin{longtable}{|p{\varnamewidth}|p{\vardescrwidth}|l}
\cline{1-2}
\cline{1-2} \centering \textbf{Name} & \centering \textbf{Description}& \\
\cline{1-2}
\endhead\cline{1-2}\multicolumn{3}{r}{\small\textit{continued on next page}}\\\endfoot\cline{1-2}
\endlastfoot\multicolumn{2}{|l|}{\textit{Inherited from object}}\\
\multicolumn{2}{|p{\varwidth}|}{\raggedright \_\_class\_\_}\\
\cline{1-2}
\end{longtable}

    \index{hal \textit{(package)}!hal.maths \textit{(package)}!hal.maths.maths \textit{(module)}!hal.maths.maths.EightQueen \textit{(class)}|)}
    \index{hal \textit{(package)}!hal.maths \textit{(package)}!hal.maths.maths \textit{(module)}|)}
