%
% API Documentation for PyHal
% Package hal.ml
%
% Generated by epydoc 3.0.1
% [Fri Jan 20 11:54:29 2017]
%

%%%%%%%%%%%%%%%%%%%%%%%%%%%%%%%%%%%%%%%%%%%%%%%%%%%%%%%%%%%%%%%%%%%%%%%%%%%
%%                          Module Description                           %%
%%%%%%%%%%%%%%%%%%%%%%%%%%%%%%%%%%%%%%%%%%%%%%%%%%%%%%%%%%%%%%%%%%%%%%%%%%%

    \index{hal \textit{(package)}!hal.ml \textit{(package)}|(}
\section{Package hal.ml}

    \label{hal:ml}

%%%%%%%%%%%%%%%%%%%%%%%%%%%%%%%%%%%%%%%%%%%%%%%%%%%%%%%%%%%%%%%%%%%%%%%%%%%
%%                                Modules                                %%
%%%%%%%%%%%%%%%%%%%%%%%%%%%%%%%%%%%%%%%%%%%%%%%%%%%%%%%%%%%%%%%%%%%%%%%%%%%

\subsection{Modules}

\begin{itemize}
\setlength{\parskip}{0ex}
\item \textbf{data}
  \textit{(Section \ref{hal:ml:data}, p.~\pageref{hal:ml:data})}

  \begin{itemize}
\setlength{\parskip}{0ex}
    \item \textbf{parser}: Parsers for raw databases. 


  \textit{(Section \ref{hal:ml:data:parser}, p.~\pageref{hal:ml:data:parser})}

  \end{itemize}
\item \textbf{features}: Collection of methods to find weights of features and select the best ones. 


  \textit{(Section \ref{hal:ml:features}, p.~\pageref{hal:ml:features})}

\item \textbf{models}
  \textit{(Section \ref{hal:ml:models}, p.~\pageref{hal:ml:models})}

  \begin{itemize}
\setlength{\parskip}{0ex}
    \item \textbf{classification}: Prediction methods based on classification algorithms. 


  \textit{(Section \ref{hal:ml:models:classification}, p.~\pageref{hal:ml:models:classification})}

    \item \textbf{pipelined}: Prediction methods based on multiple models mixed up. 


  \textit{(Section \ref{hal:ml:models:pipelined}, p.~\pageref{hal:ml:models:pipelined})}

    \item \textbf{regression}: Prediction methods based on regression algorithms. 


  \textit{(Section \ref{hal:ml:models:regression}, p.~\pageref{hal:ml:models:regression})}

    \item \textbf{time\_series}: Multi-purpose prediction methods to be used in time-series. 


  \textit{(Section \ref{hal:ml:models:time_series}, p.~\pageref{hal:ml:models:time_series})}

  \end{itemize}
\item \textbf{predict}: " General model to make prediction about everything. 


  \textit{(Section \ref{hal:ml:predict}, p.~\pageref{hal:ml:predict})}

\item \textbf{utils}: Various tools and utilities to deal with database and machine learning. 


  \textit{(Section \ref{hal:ml:utils}, p.~\pageref{hal:ml:utils})}

\end{itemize}


%%%%%%%%%%%%%%%%%%%%%%%%%%%%%%%%%%%%%%%%%%%%%%%%%%%%%%%%%%%%%%%%%%%%%%%%%%%
%%                               Variables                               %%
%%%%%%%%%%%%%%%%%%%%%%%%%%%%%%%%%%%%%%%%%%%%%%%%%%%%%%%%%%%%%%%%%%%%%%%%%%%

  \subsection{Variables}

    \vspace{-1cm}
\hspace{\varindent}\begin{longtable}{|p{\varnamewidth}|p{\vardescrwidth}|l}
\cline{1-2}
\cline{1-2} \centering \textbf{Name} & \centering \textbf{Description}& \\
\cline{1-2}
\endhead\cline{1-2}\multicolumn{3}{r}{\small\textit{continued on next page}}\\\endfoot\cline{1-2}
\endlastfoot\raggedright \_\-\_\-p\-a\-c\-k\-a\-g\-e\-\_\-\_\- & \raggedright \textbf{Value:} 
{\tt None}&\\
\cline{1-2}
\end{longtable}

    \index{hal \textit{(package)}!hal.ml \textit{(package)}|)}
